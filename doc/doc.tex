\documentclass[a4paper,12pt]{article}

\usepackage[utf-8]{inputenc}
\usepackage[czech]{babel}

\author{Radek Podgorný}
\title{Operační a dohledové centrum telefonní ústředny EWSD}
\date{}

\usepackage{setspace}

\begin{document}

\maketitle

\section{Architektura}

\subsection{Dekompozice programu}

Operacni a dohledove centrum systemu Siemens EWSD tvori dvojice programu
\emph{ewrecv} a \emph{ewterm}. Volitelne muze byt jeste pouzit program
\emph{ewalarm} pro sber alarmu.

Program \emph{ewrecv} zajistuje serverovou
cast komunikace a bezi typicky v jedine instanci na centralnim pocitaci.
Udrzuje nekolik X.25 spojeni (po jednom do kazde ustredny), pres nez uskutecnuje
vlastni povelovani a sbirani odpovedi a alarmu. Tato spojeni se snazi
neblokujicim zpusobem udrzovat stale otevrena. Na opacne strane udrzuje
spojeni s nekolika instancemi programu \emph{ewterm}, pres ktera prijima povely
a odesila odpovedi a alarmy.

Program \emph{ewterm} je klientska cast celeho systemu a umoznuje uzivateli
zadavat prikazy jednotlivym ustrednam a sledovat odpovedi. Volitelne je mozne
zaregistrovat se k odebirani alarmu.

Program \emph{ewalarm} je tou nejjednodussi moznou verzi klientske implementace
a jeho jedinym ucelem je pripojeni na \emph{ewrecv} a nesledne odebirani alarmu.
Tato poplachova hlaseni odesila na standardni vystup. Jak uz podstaty vyplyva,
je tento program neinteraktivni.

Komunikace mezi \emph{ewrecv} a \emph{ewterm} je zalozena na umele definovanem
stavovem protokolu \emph{IProto}, ktery je velmi podrobne
zdokumentovan v souboru \emph{iproto.txt}. Komunikace s ustrednou pak probiha
pres proprietarni protokol \emph{DIALOG} (nazev prejat od firmy Siemens).

\subsection{Realizace komunikace na vrstve X.25}

Vzhledem k tomu, ze fyzicky neni spojeni s ustrednami realizovano pres X.25
linku, je nutne obalovat pakety protokolu X.25 do protokolu XOT podle standardu
firmy Cisco. Protokol XOT je specifikovan v RFC 1613. Vlastni program
\emph{ewrecv} pouziva X.25 sockety (protokol X.25 je implementovan v linuxovem
jadre), ale vzhledem k tomu, ze linuxove jadro neobsahuje podporu protokolu
XOT, je nutne resit komunikacni retezec nasledujicim zpusobem. Jaderny modul
\emph{x25tap} se dynamicky nacte do jadra a vytvori virtualni fyzicke X.25
zarizeni, pres ktere probiha komunikace. Ve skutecnosti ale pakety predava
do uzivatelskeho prostoru, kde je zpracuje demon \emph{xotd}. Ten se postara
o samotne obaleni a pakety znovu posle, tentokrat vsak pres standardni TCP
socket. Stejna vymena probiha samozrejme i v opacnem smeru.

Z vyse popsaneho vyplyva, ze pro funkcnost komunikace je nutne mit nejprve
zaveden modul \emph{x25tap} a spusteneho demona \emph{xotd}.

\section{Popis jednotlivych programu}

\subsection{ewrecv}

Program \emph{ewrecv} je srdcem celeho systemu. Realizuje multiplexni pristup
smerem k uzivatemum (nekolik soucasne pripojenych \emph{ewtermu}u) i smerem
k ustrednam (soubezne pripojeni na vice ustreden). Diky tomuto obousmernemu
multiplexingu je mozne, aby uzivatel poveloval vice ustreden soucasne i aby
vice uzivatelu prijimalo alarmy od jedine ustredny (samozrejme lze realizovat
kombinaci, kdy vsichni uzivatele prijimaji alarmy od vsech ustreden).

Program \emph{ewrecv} je nutne spoustet s pravy roota, protoze pouziva zamky.

Nasleduje prehled nejdulezitejsich argumentu, ktere je mozne (nekdy nutne)
zadat pri spousteni na prikazove radce:

\begin{itemize}
\item \verb!-h! -- Zobrazi podrobnou napovedu k jednotlivym argumentum.
\item \verb!--x25local! -- Pomoci tohoto argumentu specifikujeme X.25 adresu
lokalniho uzlu. Tato musi samozrejme odpovidat nastaveni routingu v siti.
\item \verb!--x25remote! -- Pomoci tohoto argumentu specifikujeme jmena a
X.25 adresy jednotlivych ustreden, na ktere chceme navazat spojeni. Format je
\verb!USTR1=ADR1:USTR2=ADR2:...!
\item \verb!--mlog! -- Tento argument nam umoznuje zadat jmeno souboru, do
ktereho chceme ukladat veskerou uzivatelskou komunikaci s ustrednami. Slouzi
jako zdroj informaci pro eventualni dohledani historie.
\item \verb!--alog! -- Podobny argument jako \verb!--mlog!, ale tentokrat
specifikujeme jmeno souboru, kam se maji ukladat veskere prijate alarmy.
Diky existenci programu \emph{ewalarm} je teoreticky tato volba zbytecna,
ale je nutne si uvedomit, ze se jedna o mnohem jistejsi zpusob zaznamenavani
alarmu, nebot v pripade padu/ukonceni programu \emph{ewalarm} by doslo
ke ztratam. Volba \verb!--alog! nam poslytuje spolehlivy zpusob zaznamu, zavisly
pouze na behu samotneho programu \emph{ewrecv}.
\item \verb!-g! -- Velmi uzitecnou moznosti je zakaz volani \verb!fork()!
a ponechani programu \emph{ewrecv} "na popredi". Takoveto chovani vyvola
prave tento argument.
\end{itemize}

\subsection{ewterm}

Program \emph{ewterm} je mozne (a z bezpecnostnich duvodu i zadouci) spoustet
pod beznym uzivatelskym uctem. Nepouziva zamykani, ani neotevira naslouchajici
sockety.

Jeho prostredi je zalozeno na knihovne \emph{ncurses} a ovladani je
diky tomu dostatecne intuitivni.

Popis argumentu zde zamerne neuvadim, protoze v typicke konfiguraci neni
nutne zadny pouzit.
Seznam vsech moznych argumentu pro prikazovou radku i s napovedou ziskame
spustenim programu s argumentem \verb!-h!.

\subsection{ewalarm}

Program \emph{ewalarm} slouzi pouze pro sbirani alarmu z ustreden. Spoustet by se
mel, podobne jako \emph{ewterm}, pod normalnim uzivatelem.

Ihned po pripojeni na instanci programu \emph{ewrecv} zacne program prijimat
poplachova hlaseni (alarmy) a odesilat je v nezmenene forme na standardni vystup.
Diky tomu je mozne pohodlne alarmy presmerovat do souboru nebo pomoci "pipe"
do jineho programu pro dalsi automatizovani zpracovani.

Argumenty opet zamerne neuvadim, protoze ve vetsine pripadu neni nutne je uvadet.
Znamy argument \verb!-h! opet vypise nabidku moznosti s odpovidajicim popisem.

\section{Popis realizace komunikace na vrstve X.25}

\subsection{x25tap a xotd}

O duvodech nutnosti pouziti modulu \emph{x25tap} a demona \emph{xotd} jsem
se zminil jiz v uvodu. V teto casti si popiseme konkretni kroky, potrebne
k uspesne realizaci napojeni na X.25 sit.

Modul \emph{x25tap} se zavede pomoci standardnich mechanismu, napriklad utilitou
\emph{insmod} nebo \emph{modprobe} a to samozrejme pod uzivatelem "root".

Dalsim podstatnym krokem je nastaveni smerovacich tabulek. Vzhledem k tomu, ze
nevyzadujeme slozitejsi nastaveni, postaci nastavit ve vrstve X.25 neco, co by
se dalo v protokolu IP nazvat "default route".

\makebox[\textwidth]{}
\verb!ifconfig x25tap0 up! \linebreak
\verb!route add --x25 0/0 x25tap0! \linebreak
\makebox[\textwidth]{}

Nasledne je nutne spustit demona \emph{xotd} pomoci

\makebox[\textwidth]{}
\verb!xotd -v x25tap0 <ip_addr>! \linebreak
\makebox[\textwidth]{}

kde \verb!<ip_addr>! je IP adresa routeru,
ktery realizuje branu do X.25 site. Argument \verb!-v! spusti demona ve
"verbose" rezimu, kdy vypisuje velice podrobne informace o pribihajicich
spojenich a posilanych paketech. Tato volba je velmi uzitecna pri pocatecni
instalaci, ale pro vysledny provoz je mozne ji po dostatecnem odladeni nepouzit.

\end{document}
